\documentclass[letter]{article}

\usepackage{amsmath}
\usepackage{graphicx}
\usepackage{geometry}
\usepackage{braket} %Can do bra-ket notation with \braket{}
\usepackage{framed} %Adds the framed environment
\usepackage{fancyhdr}
\usepackage{datetime} %For formatting of header date
\usepackage{ulem} %Makes strike-through lines with \sout{}
\usdate %Month, Dth, YYYY
\geometry{
  letterpaper,
  left=1in,
  right=1in,
  bottom=1in,
  top=1in}
\pagestyle{fancy}
\lhead{NE101 Midterm 2 Andrew's Review Session}
\chead{}
\rhead{}
\lfoot{}
\cfoot{\thepage}
\rfoot{\today \quad \currenttime}
\setlength\parindent{0pt}

\begin{document}
\textbf{\Large{Andrew's Review Session}} \\
\vspace{12pt}
%\cite[pp. 45]{krane}
%\cite[Lec 24]{lecture}


\section{Nuclear Reactions}

\begin{equation}
R= \sigma \phi N
\end{equation}

\begin{itemize}
\item \textbf{cross section}: measure of the relative probability for the reaction to occur; units of b= 10$^{-24}$cm$^{2}$
\item $\sigma$ used to determine activity 
\end{itemize}

\textbf{Kinematics}


$$X(a,b)Y$$
$$Q= (m_{X}+m_{a}-m_{Y}-m_{b})c^{2}$$
$$Q =T_{Y}+T_{b}-T_{X}-T_{a}$$

\begin{itemize}
\item Exothermic Q$>$0 : \\$m_{X}+m_{a}>m_{Y}+m_{b}$,\\ $T_{Y}+T_{b}>T_{X}+T_{a}$
\item Endothermic Q$<$0 : \\$m_{X}+m_{a}<m_{Y}+m_{b}$,\\ $T_{Y}+T_{b}<T_{X}+T_{a}$
\item Ultimately, equation 11.5 from Krane
\item reaction reaches excited states of Y: \\ $Q_{ex} = (m_{X}+m_{a}-m_{Y*}-m_{b})c^{2}$\\$Q_{ex} = Q_{0}-E_{ex}$
\item compound nucleus \\ $Q = -T_{a} = (m_{X}+m_{a}-m_{C*})c^{2}-E_{ex}$
\end{itemize}

Population of an excited state of a daughter nucleus by alpha or beta decay can then de-excite by gamma-decay or IC.

\section{Nuclear Structure}

What do the models predict? E(4+)/E(2+), Q(2+), energy levels, spin parity assignments

\vspace{10pt}

What mass region is a particular model the best description? Vibration: A$<$150, Rotation 150$<$A$<$190 and A$>$230 

Depending on the model, what is the size of the gap between the ground state and the first excited state? 

\vspace{10pt}

Shell Model vs. Collective Model ("liquid drop")

\begin{itemize}
\item shell model: a blend of infinite well and SHO\\ 
\hspace{10pt} single particle (a single valence proton or neutron) \\
\hspace{10pt} multi-particle (unpaired protons or neutrons)\\
\hspace{10pt} magic numbers: 2, 8, 20, 28, 50, 82, 126, 184
\vspace{10pt}
\item Vibration Model (collective)
\vspace{10pt}
\item Rotational Model (collective)
\end{itemize}

\section{Decay Modes}

What makes a decay mode preferred over another? Which nuclei are more likely to go through a particular decay?

\vspace{10pt}

For all decay modes, transitions with the least change in angular momentum are preferred

\subsection{$\alpha$-Decay}
\begin{itemize}
\item \begin{equation}
Q = (m_{X}-m_{X'}-m_{\alpha})c^{2} = T_{X'}-T_{\alpha}=B(\alpha)+B(X')-B(X)
\end{equation}
\item preferential for removing 5-8 MeV through decrease of mass (mode preferred by have heavy nuclei)

\item a Coulomb repulsion effect, strong force

\item emission of $\alpha$ is preferred over other particles is because it is the most energy efficient, B/A

\item large disintegration energies (Q) had short half-lives; adding neutrons to a nucleus reduces the disintegration energy

\item occurs through tunneling

\item half life increases with increasing Coulomb Barrier\\
\hspace{20pt} How do you increase the Coulomb Barrier? Example: carrying away a larger amount of angular momentum

\item hindrance: similarity of the initial and final wave functions

\item permitted values of $l_{\alpha}$

\item discrete energy spectrum

\end{itemize}

\subsection{$\beta$-Decay}
\begin{itemize}

\item method for sliding down the mass parabola

\item represents the drive to reach a more optimal charge ratio

\item drawing a relation to the Semi Empirical Mass Formula, ways to increase the Binding Energy (i.e. more stable configuration):\\
\hspace{20pt}  increase symmetry $\rightarrow$ decreases $a_{sym}(A-2Z)^{2}/A$\\
\hspace{20pt}  decrease proton repulsion\\
\hspace{20pt}  increase pairing

\item $\beta+$ decay and electron capture have thresholds, 2m$_{e}$c$^{2}$ and B$_{n}=13.6eV\times\frac{Z^{2}}{n^{2}}$, respectively. In addition,  these two decays compete. 

\item Selection Rules: which is preferred and why? 

\item continuous energy spectrum (exception: EC results in a monoenergetic neutrino)

\end{itemize}

\subsection{$\gamma$-Decay}
\begin{itemize}

\item preferred mode for carrying away angular momentum from excited states (rearrangement of nucleons into a more stable spatial arrangement)

\item Selection Rules: Which transition are possible?

\item WE (based on single particle model) conclusions

\item Gamma-decay and IC compete\\
\hspace{20pt} IC is a low energy event (decreasing transition energy)\\
\hspace{20pt} large change in J results in more probable IC\\
\hspace{20pt} IC is more important for heavy nuclei
\hspace{20pt} 0+ to 0+ transition occurs by IC

\item Compton scattering characteristics: relationship between electron shells, their binding energy, and absorption cross section

\end{itemize}

\section{General }

\begin{itemize}

\item half-life trends

\item selection rules

\item conservation of energy and momentum

\end{itemize}

\end{document}
